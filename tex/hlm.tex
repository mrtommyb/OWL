\documentclass[12pt,letterpaper]{article}

\newcommand{\foreign}[1]{\textsl{#1}}
\newcommand{\project}[1]{\textsl{#1}}

\begin{document}

Precise photometry is the bread-and-butter of the exoplanet community.
It is also crucial for [other things].

Imagine you are trying to do very sensitive relative photometry on a variable source.
The most precise methodologies at present involve fixing the position of the source on the detector
  (cite examples)
  to reduce dependency of the results on the flat-field or flat-fielding errors.
They also often involve defocusing the telescope
  (cite examples),
  to illuminate the pixels more uniformly
  (and hence reduce intra-pixel sensitivity issues),
  avoid saturation,
  and make the observations less atmosphere-dependent.
There are also strategies related to calibration or the choices of comparison stars
  (cite examples).
Once all these choices are made,
  the choice of methodology for actually photometering the source remains.
Most projects choose something akin to aperture photometry
  (cite examples),
  in which the pixels within an aperture are co-added with unit weights,
  and pixels outside the aperture are not used at all.
Here we ask only about this last choice:
What is the best way to measure the brightness of a nearly constant source,
  given a set of precise images of the source taken over time?

We are going to answer this question using optimization, of course;
  we are going to find the ``best'' method.
We will assume that the investigator obeyed the usual rules,
  so he or she fixed the position of the star on the detector,
  defocused the telescope,
  and avoided saturation.
However, we are also going to assume---%
  as is \emph{always the case}---%
  that there are some residual,
  unaccounted-for positional offsets (jitter) and point-spread-function changes.
These offsets and changes could be random or systematic over time,
  but we will assume that they are unknown \foreign{a priori}.
That is, the telescope house-keeping data are not good enough to track them at the level we care about.
We are trying to measure \emph{extremely} precise photometry,
  at the level of $10^{-4}$ or $10^{-5}$ or better.

The conditions we have assumed might seem strange,
  but they are generic for the most precise photometry systems currently operating.
\project{Kepler}, for instance, has all of these problems,
  as does [insert projects here, like Wright's].

% in what follows the object of photometry is a ``star'' not a ``source''.

You can think of this problem as the problem of finding the best strategy in a single-player game.
The rules of this game are as follows:
\begin{itemize}
\item
  You have multi-epoch single-band imaging (many images)
  and a position in that imaging of a star you care about (the ``object'' star).
  You might also have some ``pixel mask'' indicating what pixels you are permitted to use
  in the vicinity of that star.
\item
  You may or may not have a list of ``comparison'' stars
  that are used to calibrate or ratio the photometry of the object star.
  If you don't, then you can assume that the imaging is very well calibrated.
\item
  You expect the star to be fairly stationary,
  but that in each image it has been offset slightly by some amount from its fiducial position.
  You have no meta-data to help you understand those small offsets.
  You also assume that any comparison stars are subject to small offsets.
  The comparison-star offsets are not necessarily the same as the object-star offsets,
  because there is camera rotation, variable optical distortion, proper motion, and parallax.
\item
  You expect the point-spread function (PSF) to be fairly constant,
  but that in each image it has a small difference from the fiducial point-spread function.
  Again, you have no meta-data to help with this,
  and you can't assume that the comparison star PSFs vary the same way.
\item
  You don't fully believe any meta-data you have about read noise or gains,
  so you don't have a precise noise estimate for any pixel at any epoch.
\item
  When you photometer a star,
  the only operation you are permitted to do is make a linear weighted sum of pixels in the images.
  Furthermore, this weighted sum must be precisely identical from epoch to epoch.
  That is, the ``aperture''
  (which in fact will be some complicated set of pixel weights)
\end{itemize}
These rules are generic.  Etc.

\end{document}
